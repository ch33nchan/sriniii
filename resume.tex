\documentclass[a4paper,10pt]{article}

% Package imports
\usepackage{latexsym}
\usepackage{fontawesome5}
\usepackage{xcolor}
\usepackage{float}
\usepackage{xcolor}
\usepackage{ragged2e}
\usepackage[empty]{fullpage}
\usepackage{wrapfig}
\usepackage{lipsum}
\usepackage{tabularx}
\usepackage{titlesec}
\usepackage{geometry}
\usepackage{marvosym}
\usepackage{verbatim}
\usepackage{enumitem}
\usepackage{fancyhdr}
\usepackage{multicol}
\usepackage{graphicx}
\usepackage{cfr-lm}
\usepackage[T1]{fontenc}
\usepackage{fontawesome5}
\usepackage{comment}
% Color definitions
\definecolor{darkblue}{RGB}{0,0,139}

% Page layout
\setlength{\multicolsep}{0pt}
\pagestyle{fancy}
\fancyhf{} % clear all header and footer fields
\fancyfoot{}
\renewcommand{\headrulewidth}{0pt}
\renewcommand{\footrulewidth}{0pt}
\geometry{left=1.4cm, top=0.8cm, right=1.2cm, bottom=1cm}
\setlength{\footskip}{5pt} % Addressing fancyhdr warning

% Hyperlink setup (moved after fancyhdr to address warning)
\usepackage[hidelinks]{hyperref}
\hypersetup{
    colorlinks=true,
    linkcolor=darkblue,
    filecolor=darkblue,
    urlcolor=darkblue,
}

% Custom box settings
\usepackage[most]{tcolorbox}
\tcbset{
    frame code={},
    center title,
    left=0pt,
    right=0pt,
    top=0pt,
    bottom=0pt,
    colback=gray!20,
    colframe=white,
    width=\dimexpr\textwidth\relax,
    enlarge left by=-2mm,
    boxsep=4pt,
    arc=0pt,outer arc=0pt,
}

% URL style
\urlstyle{same}

% Text alignment
\raggedright
\setlength{\tabcolsep}{0in}

% Section formatting
\titleformat{\section}{
  \vspace{-4pt}\scshape\raggedright\large
}{}{0em}{}[\color{black}\titlerule \vspace{-7pt}]

% Custom commands
\newcommand{\resumeItem}[2]{
  \item{
    \textbf{#1}{\hspace{0.5mm}#2 \vspace{-0.5mm}}
  }
}

\newcommand{\resumePOR}[3]{
\vspace{0.5mm}\item
    \begin{tabular*}{0.97\textwidth}[t]{l@{\extracolsep{\fill}}r}
        \textbf{#1}\hspace{0.3mm}#2 & \textit{\small{#3}}
    \end{tabular*}
    \vspace{-2mm}
}

\newcommand{\resumeSubheading}[4]{
\vspace{0.5mm}\item
    \begin{tabular*}{0.98\textwidth}[t]{l@{\extracolsep{\fill}}r}
        \textbf{#1} & \textit{\footnotesize{#4}} \\
        \textit{\footnotesize{#3}} &  \footnotesize{#2}\\
    \end{tabular*}
    \vspace{-2.4mm}
}

\newcommand{\resumeProject}[4]{
\vspace{0.5mm}\item
    \begin{tabular*}{0.98\textwidth}[t]{l@{\extracolsep{\fill}}r}
        \textbf{#1} & \textit{\footnotesize{#3}} \\
        \footnotesize{\textit{#2}} & \footnotesize{#4}
    \end{tabular*}
    \vspace{-2.4mm}
}

\newcommand{\resumeSubItem}[2]{\resumeItem{#1}{#2}\vspace{-4pt}}

\renewcommand{\labelitemi}{$\vcenter{\hbox{\tiny$\bullet$}}$}
\renewcommand{\labelitemii}{$\vcenter{\hbox{\tiny$\circ$}}$}

\newcommand{\resumeSubHeadingListStart}{\begin{itemize}[leftmargin=*,labelsep=1mm]}
\newcommand{\resumeHeadingSkillStart}{\begin{itemize}[leftmargin=*,itemsep=1.7mm, rightmargin=2ex]}
\newcommand{\resumeItemListStart}{\begin{itemize}[leftmargin=*,labelsep=1mm,itemsep=0.5mm]}

\newcommand{\resumeSubHeadingListEnd}{\end{itemize}\vspace{2mm}}
\newcommand{\resumeHeadingSkillEnd}{\end{itemize}\vspace{-2mm}}
\newcommand{\resumeItemListEnd}{\end{itemize}\vspace{-2mm}}
\newcommand{\cvsection}[1]{%
\vspace{2mm}
\begin{tcolorbox}
    \textbf{\large #1}
\end{tcolorbox}
    \vspace{-4mm}
}

\newcolumntype{L}{>{\raggedright\arraybackslash}X}%
\newcolumntype{R}{>{\raggedleft\arraybackslash}X}%
\newcolumntype{C}{>{\centering\arraybackslash}X}%

% Commands for icon sizing and positioning
\newcommand{\socialicon}[1]{\raisebox{-0.05em}{\resizebox{!}{1.2em}{#1}}}
\newcommand{\ieeeicon}[1]{\raisebox{-0.3em}{\resizebox{!}{1.3em}{#1}}}

% Font options
\newcommand{\headerfonti}{\fontfamily{phv}\selectfont} % Helvetica-like (similar to Arial/Calibri)
\newcommand{\headerfontii}{\fontfamily{ptm}\selectfont} % Times-like (similar to Times New Roman)
\newcommand{\headerfontiii}{\fontfamily{ppl}\selectfont} % Palatino (elegant serif)
\newcommand{\headerfontiv}{\fontfamily{pbk}\selectfont} % Bookman (readable serif)
\newcommand{\headerfontv}{\fontfamily{pag}\selectfont} % Avant Garde-like (similar to Trebuchet MS)
\newcommand{\headerfontvi}{\fontfamily{cmss}\selectfont} % Computer Modern Sans Serif
\newcommand{\headerfontvii}{\fontfamily{qhv}\selectfont} % Quasi-Helvetica
\newcommand{\resumeheading}[4]{%
  \item[] \textbf{#1} \hfill \textit{\footnotesize{#4}} \\%
  \textit{\footnotesize{#3}} \\%
  \vspace{-2mm}%
}

\begin{document}
\headerfontiii

% Header
\begin{center}
    {\Huge\textbf{SRINIVAS}} \\[0.5em]
    +44-07553994632 \quad | \quad \href{mailto:sxt589@student.bham.ac.uk}{sxt589@student.bham.ac.uk} \quad | \quad \href{https://srinivastb.netlify.app/}{Personal Website} \\[0.5em]
    \socialicon{\faLinkedin} \href{https://www.linkedin.com/in/srinivastb/}{/srinivastb} \quad | \quad \socialicon{\faGithub} \href{https://github.com/ch33nchan}{/ch33nchan} \quad | \quad \ieeeicon{\includegraphics[height=1.3em]{google-scholar-icon.png}} \href{https://scholar.google.com/citations?hl=en&user=UgR4sMUAAAAJ}{/srinivas} \quad | \quad \socialicon{\faTwitter} \href{https://x.com/srinitwtts}{srinitwtts} \\[0.5em]
    Birmingham, United Kingdom
\end{center}


\vspace{1mm}

\section{\textbf{About Me}}
\vspace{1mm}
\small{
\begin{itemize}
    \item Machine Learning Engineer working on reinforcement learning and robotics, developing intelligent systems for real-world applications.
    
    \item Creating simulation environments and training pipelines to bridge AI research with robotics, experimenting with reward engineering and multi-objective optimization for robotic behavior.
    
    \item Implementing reinforcement learning algorithms such as PPO and SAC for robot control tasks, improving training efficiency and performance.
    
    \item Building software systems that integrate perception, planning, and control, applying software architecture principles to maintainable machine learning code.
\end{itemize}
}
\vspace{-2mm}
\section{\textbf{Technical Skills}}

\textbf{Programming Languages}:  
Python, C++, SQL, Bash, CMake
\vspace{1mm}

\textbf{Machine Learning \& AI}:  
PyTorch, TensorFlow, JAX, Scikit-Learn, XGBoost, OpenCV, Hugging Face Transformers
\vspace{1mm}

\textbf{Reinforcement Learning}:  
Stable Baselines3, Ray RLlib, OpenAI Gym, Gymnasium, Multi-Agent RL Frameworks
\vspace{1mm}

\textbf{Robotics \& Simulation}:  
ROS2, Isaac Gym, Isaac Sim, MuJoCo, Gazebo, Point Cloud Library (PCL)
\vspace{1mm}

\textbf{Data Science \& Analytics}:  
NumPy, Pandas, SciPy, Matplotlib, Seaborn, TensorBoard, Weights \& Biases
\vspace{1mm}

\textbf{MLOps \& Development}:  
Docker, Kubernetes, Ray, Horovod, MLflow, Airflow, Git, CI/CD, TensorRT, ONNX
\vspace{1mm}

\textbf{Research \& Theory}:  
Policy Gradient Methods, Actor-Critic Algorithms, Multi-Agent Coordination, Sim-to-Real Transfer
\vspace{1mm}




\section{\textbf{Education}}
\vspace{-0.4mm}
\resumeSubHeadingListStart

\resumeSubheading
{University of Birmingham}{Birmingham, United Kingdom}
{Masters in Artificial Intelligence \& Machine Learning}{Sept 2024 - Present}
\resumeItemListStart
\item \textbf{Global Masters Scholarship} – Awarded for academic excellence and leadership potential.
\item \textbf{Chancellor's Award} – Recognized for contributions to the university community.
\resumeItemListEnd

\resumeSubheading
{SRM Institute of Science \& Technology}{Chennai, India}
{Bachelor of Technology in Computer Science}{June 2024}
\resumeItemListStart
\item \textbf{First Class with Distinction} – Achieved a grade of 90.0\%.
\resumeItemListEnd

\resumeSubHeadingListEnd

\section{\textbf{Experience}}
\resumeSubHeadingListStart

\resumeSubheading
{Alan Turing Institute [\href{https://research.birmingham.ac.uk/en/projects/human-machine-teaming}{\faIcon{globe}}]}{Birmingham, United Kingdom}
{\textbf{Research Associate} | Collaboration: University of Birmingham \& U.S. Army Research Institute}{Dec 2024 - April 2025}
\vspace{0.5mm}
\resumeItemListStart
    \item Developed a \textbf{Heterogeneous-Agent Reinforcement Learning (HARL)} framework using human-proxy agents that simulate realistic human constraints and capabilities to improve AI-human collaboration in multi-agent systems.
    
    \item Designed a cooperative grid-world capture environment based on Stag Hunt game theory, where machine agents had full observability but couldn't detect target health, while human-proxy agents had limited vision but unique disease detection abilities.
    
    \item Conducted experiments across various environment configurations, varying disease probability and penalty severity to analyze cooperation patterns.
    
    \item Demonstrated that RL agents trained with human-proxy teammates achieved superior cross-environment performance, with teams trained under moderate risk conditions showing 30-40\% higher collaboration rates.

\resumeItemListEnd
\vspace{2mm}
\resumeSubheading
{RiskOpsAI [\href{https://www.optimeyes.ai/}{\faIcon{globe}}]}{San Jose, Remote}
{\textbf{AI ML Intern}}{Jun 2023 - Aug 2024}
\vspace{0.5mm}
\resumeItemListStart
  \item Built a deep learning pipeline with TensorFlow and PyTorch using ResNet-50 and Transformer models for classification. Used TensorRT to optimize inference, achieving 25\% faster performance and 15\% higher accuracy.
  
  \item Created distributed ML infrastructure with Apache Airflow and MLflow on AWS GPU clusters using Horovod, reducing training time by 20\%. Set up automated data pipelines for feature engineering.
  
  \item Developed a predictive analytics system using PostgreSQL and BigQuery with scikit-learn and XGBoost on imbalanced data. Optimized queries to improve decision-making efficiency by 30\%.
\resumeItemListEnd
\vspace{2mm}
\resumeSubheading
  {SRM Institute of Science \& Technology [\href{https://www.srmist.edu.in/faculty/vaishnavi-moorthy/}{\faIcon{globe}}]}{Chennai, India}
  {\textbf{Researcher under Dr. Vaishnavi Moorthy}}{Oct 2022 - Feb 2024}
\resumeItemListStart
  \item Developed an autonomous navigation system using ROS2 by fusing LiDAR, RGB-D, and IMU data through an Extended Kalman Filter. This improved localization accuracy by 15\%.
  \item Built a SLAM system using SAC and TRPO algorithms in PyTorch to improve path planning with RRT* and A*. Reduced navigation errors by 25\% using dynamic obstacle avoidance.
  \item Created a real-time perception pipeline using OpenCV and PCL, integrating YOLOv7 for object detection. Achieved 20ms latency and 95\% detection accuracy in changing environments.
\resumeItemListEnd

% --- Projects Section ---
% --- Projects Section ---
% --- Projects Section ---
% --- Projects Section ---
\section{\textbf{Projects}}

\resumeSubHeadingListStart

% Project 1: MADDPG
\resumeheading
  {MADDPG: Multi-Agent Deep Deterministic Policy Gradient~%
    \href{https://github.com/ch33nchan/rl-agents/tree/main/maddpg-pettingzoo-pytorch-master}{\textcolor{darkblue}{\faGithub}}
  }
  {}
  {Tools: Python, PyTorch, PettingZoo, NumPy, Multi-Agent RL}
  {Jul 2024 -- Aug 2024}

\vspace{0.5mm}

\resumeItemListStart
  \item Built a reinforcement learning system for multiple agents that can learn to work together or compete in different environments using PettingZoo.
  \item Created shared and individual policies for agents, so they can make decisions independently while still learning as a group.
  \item Added experience buffers for each agent to store past experiences and improve learning from them during training.
  \item Tested the system on tasks like simple\_adversary and simple\_spread, and tuned settings like learning rates and batch sizes for better performance.
\resumeItemListEnd

\vspace{2.5mm}

\resumeheading
  {RL2-Enhanced: Advanced Reinforcement Learning for LLMs~%
    \href{https://github.com/ch33nchan/rl2.0.1}{\textcolor{darkblue}{\faGithub}}
  }
  {}
  {Tools: Python, PyTorch, Hydra, MLflow, Weights\&Biases, Optuna, NumPy, Distributed RL}
  {Jul 2025}

\vspace{0.5mm}

\resumeItemListStart
  \item Extended the original RL2 repository by Chenmien Tan, adding advanced features for scalable and robust language model training.
  \item Implemented adaptive KL penalty mechanisms (exponential, PID, scheduled controllers) for stable PPO optimization.
  \item Developed multi-objective optimization with Pareto frontier tracking, supporting reward, entropy, and KL constraints.
  \item Integrated alternative advantage estimation methods (V-trace, Retrace($\lambda$), TD($\lambda$), multi-step returns) for improved sample efficiency.
  \item Automated hyperparameter tuning using Optuna and Hyperopt, enabling Bayesian and grid search strategies.
  \item Added advanced memory optimization: gradient checkpointing, CPU offloading, adaptive batch sizing, and detailed profiling.
  \item Enabled experiment tracking and model versioning with MLflow and Weights\&Biases for reproducible research and MLOps workflows.
\resumeItemListEnd

\vspace{2.5mm}

% Project 2: Isaac Gym Humanoid Robot


% Project 3: RL ProtoKit
\resumeheading
  {RL ProtoKit: Rapid Prototyping Toolkit for Reinforcement Learning~%
    \href{https://github.com/ch33nchan/rl-prokit}{\textcolor{darkblue}{\faGithub}}
  }
  {}
  {Tools: Python, PyTorch, Gymnasium, PettingZoo, NumPy, Reinforcement Learning}
  {Jul 2024 -- Present}

\vspace{0.5mm}

\resumeItemListStart
  \item \textbf{Developed modular CLI tool for RL prototyping} that enables quick generation of custom Gym wrappers, hyperparameter tuning, policy debugging, and full end-to-end pipelines, reducing setup time for RL experiments compared to traditional methods.
  \item \textbf{Implemented advanced RL features including prioritized replay buffers and RNN policy networks} for efficient off-policy learning and handling temporal dependencies in partially observable environments, with support for discrete and continuous action spaces.
  \item \textbf{Integrated intrinsic curiosity module and multi-agent wrappers} using PettingZoo for enhanced exploration in sparse-reward settings and cooperative multi-agent scenarios, along with frame-stacking, grayscale transforms, and Atari-specific preprocessing.
  \item \textbf{Added PPO clip annealing, KL-divergence logging, and SAC temperature auto-tuning} for stable policy optimization and entropy regularization, enabling more robust training across various RL algorithms and environments.
\resumeItemListEnd

\vspace{2.5mm}

\resumeheading
  {Isaac Gym Humanoid Robot: Teaching Robots to Walk~%
    \href{https://github.com/ch33nchan/isaac-humanoid}{\textcolor{darkblue}{\faGithub}}
  }
  {}
  {Tools: Python, PyTorch, Isaac Gym, NVIDIA GPU}
  {Nov 2023}

\vspace{0.5mm}

\resumeItemListStart
  \item Created a computer program that teaches virtual humanoid robots how to walk and balance using trial-and-error learning (reinforcement learning).
  \item Used NVIDIA's Isaac Gym simulator to run thousands of robot training sessions simultaneously on a single graphics card for faster learning.
  \item Implemented reward systems that give the robot positive feedback for good walking behavior and negative feedback for falling or poor movement.
  \item Applied PPO (Proximal Policy Optimization) algorithm to help the robot gradually improve its walking skills through repeated practice.
\resumeItemListEnd

\vspace{2.5mm}
\vspace{2.5mm}

% Project 4: TinyGrad RLCV
\resumeheading
  {TinyGrad RLCV: Lightweight RL for Computer Vision~%
    \href{https://github.com/ch33nchan/rlcv}{\textcolor{darkblue}{\faGithub}}
  }
  {}
  {Tools: Python, TinyGrad, OpenCV, NumPy, Reinforcement Learning}
  {Oct 2023 -- Dec 2023}

\vspace{0.5mm}

\resumeItemListStart
  \item Created ultra-lightweight neural network operations with TinyGrad, achieving a small memory footprint ($<$10MB) for deployment on resource-limited edge devices (mobile CPUs and ARM).
  \item Built a memory-optimized DQN agent with prioritized experience replay and flexible policy networks, enabling stable learning and real-time inference ($>$30 FPS) on target hardware.
  \item Developed an optimized OpenCV pipeline for efficient, real-time feature extraction from webcam streams, enabling robust object tracking on low-power systems.
\resumeItemListEnd

\vspace{2.5mm}

\begin{comment}
% Project 5: SAC Robot Navigation
\resumeheading
  {SAC: Soft Actor-Critic Robot Navigation~%
    \href{https://github.com/ch33nchan/isaac-sim}{\textcolor{darkblue}{\faGithub}}
  }
  {}
  {Tools: Python, PyTorch, Isaac Sim, ROS2, OpenCV}
  {Jul 2024 -- Aug 2024}

\vspace{0.5mm}

\resumeItemListStart
  \item Built a reinforcement learning system for autonomous robot navigation using Soft Actor-Critic algorithm with entropy regularization for continuous control in Isaac Sim environment.
  \item Implemented experience replay buffer system to store and sample past robot experiences for off-policy learning and improved sample efficiency during training.
  \item Created dynamic obstacle avoidance environment with 9x9 grid layout, random cube placement, and LIDAR-based state representation for robust navigation policy learning.
  \item Integrated ROS2 bridge for real-time communication between Isaac Sim and control systems, enabling seamless robot control and sensor data processing.
\resumeItemListEnd

\resumeSubHeadingListEnd

\end{comment}


\vspace{-2mm}
\section{\textbf{Publications}
\hfill \textcolor{darkblue}{\scriptsize 
C=Conference, J=Journal, P=Patent, S=In Submission, T=Thesis}}
\vspace{0.1mm}
\small{
\begin{enumerate}[leftmargin=*, labelsep=0.5em, align=left, 
widest={[\textbf{S.1}]}, itemindent=0em, label={\textbf{[\arabic*]}}]
\item[\textbf{[C.1]}] V. Moorthy, T. B. Srinivas. (2024). \href{https://doi.org/10.1007/978-981-97-1329-5_36}{\textbf{Enhancing Low-Light Surveillance Images with MirNet: A Keras-Based Approach}}. In \textit{Smart Trends in Computing and Communications}, pp. 445–458. Springer. 2024 (SmartCom Pune, India). DOI: \href{https://doi.org/10.1007/978-981-97-1329-5_36}{10.1007/978-981-97-1329-5_36}.
\item[\textbf{[C.2]}] V. Moorthy, D. Shah, S. Kapoor, S. T. B., A. Lal. (2023). \href{https://doi.org/10.1109/CONECCT58849.2023.10234740}{\textbf{ALATS: Analysis of Localization Algorithms in Terrestrial Surveillance Bots}}. In \textit{2023 IEEE International Conference on Electronics, Computing and Communication Technologies (CONECCT)}, pp. 1–6. IEEE. 2023 (Bangalore, India). DOI: \href{https://doi.org/10.1109/CONECCT58849.2023.10234740}{10.1109/CONECCT58849.2023.10234740}.
\end{enumerate}
}

\end{document}

